\clearpage
\graphicspath{{./lib/alice_incoming_data_processor/figures/}}
\section{Alice Incoming Data Processor}

\begin{tcolorbox}	
	\begin{tabular}{p{2.75cm} p{0.2cm} p{10.5cm}} 	
        \textbf{Header Files}    &:& alice\_incoming\_data\_processor\_*.h \\
		\textbf{Source Files}    &:& alice\_incoming\_data\_processor\_*.cpp \\
        \textbf{Version}         &:& 20190410 (Andoni Santos)
	\end{tabular}
\end{tcolorbox}

\maketitle
This block processes and directs the data received in the AliceQKD block.

\subsection*{Input Signals}

\begin{itemize}
	\item[0] - Binary signal with Alice's key, to be transmitted to Bob.
	\item[1] - Binary signal with Alice's basis, that will be used to encode the key.
\end{itemize}

\subsection*{Output Signals}

\begin{itemize}
	\item[0] - Binary signal with Alice's key, to be transmitted to Bob.
	\item[1] - Binary signal with Alice's basis, that will be used to encode the key.
	\item[2] - Binary signal with Alice's key, to be transmitted to Bob.
	\item[3] - Binary signal with Alice's basis, that will be used to encode the key. 
\end{itemize}

% \subsection*{Signals}

% \begin{table}[h]
% 	\centering
% 	\begin{tabular}{|c|l|}
% 		\hline
% 		\textbf{Number of Input Signals} & 2 \\ \hline
%         \textbf{Type of Input Signals} & Binary, Binary,\\\hline
%     	\textbf{Number of Output Signals} & 4 \ \\ \hline
%         \textbf{Type of Output Signals} & Binary, Binary, Binary, Binary \\ \hline
% 	\end{tabular}
% 	\caption{Alice Incoming Data Processor signals}
% 	\label{table:bin_sour_signals}
% \end{table}

\subsection*{Input Parameters}

\subsection*{Methods}

\subsection*{Functional description}
This block receives and directs the data and basis signals received by Alice. It
creates duplicate versions of the signals, so that they can be sent both to the
processing part of Alice QKD and output to the Quantum Channel. In addition, it
makes sure that these two signals are always in sync.

\subsection*{Suggestions for future improvement}