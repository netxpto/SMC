\clearpage
\graphicspath{{./lib/bob_incoming_data_processor/figures/}}
\section{Bob Incoming Data Processor}

\begin{tcolorbox}	
	\begin{tabular}{p{2.75cm} p{0.2cm} p{10.5cm}} 	
        \textbf{Header Files}    &:& bob\_incoming\_data\_processor\_*.h \\
		\textbf{Source Files}    &:& bob\_incoming\_data\_processor\_*.cpp \\
        \textbf{Version}         &:& 20190410 (Andoni Santos)
	\end{tabular}
\end{tcolorbox}

\maketitle
This block processes and directs the data received in the BobQKD block.

\subsection*{Input Signals}

\begin{itemize}
	\item[0] - TimeContinuousAmplitudeDiscreteReal signal with received key, to be transmitted to Bob.
	\item[1] - Binary signal with Bob's basis, that will be used to measure the qubits.
\end{itemize}

\subsection*{Output Signals}

\begin{itemize}
	\item[0] - TimeContinuousAmplitudeDiscreteReal signal with received key, to be transmitted to Bob.
	\item[1] - Binary signal with Bob's basis, that will be used to measure the qubits.
\end{itemize}

% \subsection*{Signals}

% \begin{table}[h]
% 	\centering
% 	\begin{tabular}{|c|l|}
% 		\hline
% 		\textbf{Number of Input Signals} & 2 \\ \hline
%         \textbf{Type of Input Signals} & TimeContinuousAmplitudeDiscreteReal, Binary,\\\hline
%     	\textbf{Number of Output Signals} & 2 \ \\ \hline
%         \textbf{Type of Output Signals} & TimeContinuousAmplitudeDiscreteReal, Binary \\ \hline
% 	\end{tabular}
% 	\caption{Bob Incoming Data Processor signals}
% 	\label{table:bin_sour_signals}
% \end{table}

\subsection*{Input Parameters}

\subsection*{Methods}

\subsection*{Functional description}
This block receives and directs the data and basis signals received by Bob. It
makes sure that these two signals are always in sync and controls the flow of data.

\subsection*{Suggestions for future improvement}