\clearpage
\graphicspath{{./lib/message_processors/figures/}}
\section{Message Processors}

\begin{tcolorbox}	
	\begin{tabular}{p{2.75cm} p{0.2cm} p{10.5cm}} 	
        \textbf{Header Files}    &:& message\_processors\_common\_*.h \\
                                &:& message\_processor\_transmitter\_*.h \\
                                &:& message\_processor\_receiver\_*.h \\
		\textbf{Source Files}    &:& message\_processors\_common\_*.cpp \\
                                &:& message\_processor\_transmitter\_*.cpp \\
                                &:& message\_processor\_receiver\_*.cpp \\
        \textbf{Version}        &:& 20190410 (Andoni Santos)
	\end{tabular}
\end{tcolorbox}

\maketitle
These blocks are used to send and receive well defined sets of data in organized
groups. They are meant to be used as a means of classically transmitting data
between two parties. Currently these allow doing all the necessary classical
communication needed to implement the BB84 protocol, transmitting discrete sets
of data of arbitrary size and type at once.

The \textit{MessageProcessorTransmitter} block is used to convert the data it
receives into a Message structure, and send it on an appropriate Message signal.
The \textit{MessageProcessorReceiver}, on the other hand, is used to receive
Messages from appropriate signals, convert from the structure into numeric data,
and output the data into the appropriate signal, according to its type.

The \textit{message\_processors\_common} files don't contain a particular block.
Instead, they have auxiliary functions, variables and types which are useful
working with messages and the message processors. Both the transmitter and
receiver make use of these files, but other blocks may use resort to them in
order to get access to useful functions.

\subsection*{Signals}

\subsubsection*{Message Processor Transmitter}
\begin{table}[h]
    \centering
	\begin{tabular}{|c|l|}
		\hline
		\textbf{Number of Input Signals} & 6 \\ \hline
        \textbf{Type of Input Signals} & TimeDiscreteAmplitudeContinuousReal \\ \hline
    	\textbf{Number of Output Signals} & 1 \ \\ \hline
        \textbf{Type of Output Signals} & Message \\ \hline
    \end{tabular}
    \label{table:mpt_signals}
\end{table}

\subsubsection*{Message Processor Receiver}
\begin{table}[h]
    \centering
    \begin{tabular}{|c|l|}
        \hline
         \textbf{Number of Input Signals} & 1 \\ \hline
        \textbf{Type of Input Signals} & Message \\ \hline
        \textbf{Number of Output Signals} & 6 \ \\ \hline
        \textbf{Type of Output Signals} & TimeDiscreteAmplitudeContinuousReal \\ \hline
    \end{tabular}
    \label{table:mpr_signals}
\end{table}

\subsection*{Input Parameters}

No input parameters.

\subsection*{Methods}

\subsubsection*{Message Processor Transmitter}
\bigbreak
MessageProcessorReceiver(initializer\_list<Signal*> InputSig, initializer\_list<Signal*> OutputSig)
\bigbreak	
void initialize(void);
\bigbreak	
bool runBlock(void);
\bigbreak

\subsubsection*{Message Processor Receiver}
\bigbreak
MessageProcessorTransmitter(initializer\_list<Signal*> InputSig, initializer\_list<Signal*> OutputSig)
\bigbreak
void initialize(void);
\bigbreak
bool runBlock(void);
\bigbreak

\subsubsection*{Message Processor Common}
\bigbreak
t\_message\_data MessageProcessors::getMessageData( const t\_message\& msg,
t\_message\_data\_length dataLength);
\bigbreak
string MessageProcessors::generateMessageData(const t\_message\_data \&msgData,
const t\_message\_type \&msgType);
\bigbreak
t\_message\_data\_length MessageProcessors::getMessageDataLength(const t\_message\&
msg);
\bigbreak
t\_message\_type MessageProcessors::getMessageType(const t\_message\& msg);
\bigbreak
bool MessageProcessors::sendToMsgProc(vector <t\_integer>\& data, Signal\&
signalToMsg, bool\& started);
\bigbreak
bool MessageProcessors::recvFromMsgProc(vector <t\_integer>\& data, Signal\&
signalToMsg, bool\& started, int\& numberofValuesToRecv);
\bigbreak

\subsection*{Functional description}

In the case that any of the signals in question is not required, a signal
leading to nowhere may be used both in the transmitter and the receiver to
bypass using that signal. That may allow using only later types of messages.

\subsubsection*{Message Processor Transmitter}
When run, the \textit{MessageProcessorTransmitter} verifies if there is anything
in any of its input signals. If there is, it starts getting values $n+1$ values
from the buffer, where $n$ is the first value it receives. If $n$ is bigger than
the buffer, it continues to retrieve values in subsequent runs until it has all
the values. When it finally has all the values to send in the message, it builds
the message, with its structure depending on which signal were the values
retrieved from.

Each message is a structure containing three \textit{string} fields:
\textit{messageType}, \textit{messageDataLength}, and \textit{messageData}. The
first defines the type of information that the message carries. The possible
types are defined in the \text{MessageProcessorsCommon} header. The second
indicates the data string length, and the third carries the data itself, in
string form. 

Afterwards, it sends the message through the output signals.
Each time it is run, it does this for all the input signals, and can send as
many messages at once as it has signals.
In order for blocks to send data to the message processor, the method
\textit{MessageProcessors::sendToMsgProc} from the
\textit{MessageProcessorsCommon} can be used. This method provides the
functionality to send $n+1$ values to the buffer, where the first value is the
total number of values to send. In addition, if the buffer it too small for all
the values, it continues to do so after it is has more space, until it has sent
all the relevant values.

At this moment, the signals index and type correspondence is fixed as follows:

\begin{itemize}
    \item 0 - BasisReconciliationMsg
    \item 1 - ErrorCorrectionMsg
    \item 2 - ErrorCorrectionPermutationsMsg
    \item 3 - PrivacyAmplificationSeedsMsg
    \item 4 - ParameterEstimationSeedAndBitsMsg
    \item 5 - ToggleBERChangeMsg
\end{itemize}

\subsubsection*{Message Processor Transmitter}
This block does the inverse work: it checks the input \textit{Message} signal to
see if any message has arrived. If it has, it retrieves it, identifies its type,
extracts its information and outputs the data to the relevant signal, preceded
by a total count of the number of values. Again, like in the
\textit{MessageProcessorTransmitter} block, the data can be read using the
\textit{MessageProcessors::recvFromMsgProc} method from the
\textit{MessageProcessorsCommon} files. This allows receiving more data than the
buffer can carry at any one time.

At this moment, the signals index and type correspondence is fixed as follows:
\begin{itemize}
    \item 0 - BasisReconciliationMsg
    \item 1 - ErrorCorrectionMsg
    \item 2 - ErrorCorrectionPermutationsMsg
    \item 3 - PrivacyAmplificationSeedsMsg
    \item 4 - ParameterEstimationSeedAndBitsMsg
    \item 5 - ToggleBERChangeMsg
\end{itemize}

\subsubsection*{Current Message Types:}

Currently, the following message types are implemented, each required by a block
needed by the BB84 protocol:

\begin{itemize}
    \item BasisReconciliationMsg - this messages carry the information for Alice
    and Bob to do Basis reconciliation. All the values are integers between 0 and
    5.
    \item ErrorCorrectionMsg - this messages carry information for the cascade
    process. All the transmitted data is composed of integers between 0 and 5.
    \item ErrorCorrectionPermutationsMsg - this messages carry the seed for
    generating the random permutations used to scramble the data in sequential
    cascade iterations. It should contain a single large integer.
    \item PrivacyAmplificationSeedsMsg - this message carries seeds used for
    hashing the key during error correction. It should contain one or more large
    integers.
    \item ParameterEstimationSeedAndBitsMsg - this message carries the seeds for
    selecting random bits from the data in order to estimate the bit error rate.
    In addition, it also carries the values of said bits. It is composed of a
    single large integer followed by binary values.
    \item ToggleBERChangeMsg - this message signals to Alice when Bob updates
    the Cascade parameters based on a new BER estimation. It contains a single "1".
\end{itemize}

\subsection*{Suggestions for future improvement}

Currently, the number and type of signals is fixed and cannot be easily changed
adapted. In the future, this could be modified to allow adapting the
message processors to situations other than the BB84 protocol. In order to do
that, it should be possible to choose a given number of signals and define the
desired type for each of them.

